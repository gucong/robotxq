\documentclass[a4paper]{article}
\usepackage{xeCJK}
\setCJKmainfont[BoldFont=SimHei,ItalicFont=KaiTi_GB2312]
{SimSun}
\setCJKsansfont{SimHei}
\setCJKmonofont{WenQuanYi Micro Hei Mono}
\setCJKfamilyfont{zhsong}{SimSun}
\setCJKfamilyfont{zhhei}{SimHei}
\setCJKfamilyfont{zhkai}{KaiTi_GB2312}
\setCJKfamilyfont{zhfs}{FangSong_GB2312}
\newcommand*{\songti}{\CJKfamily{zhsong}}
\newcommand*{\heiti}{\CJKfamily{zhhei}}
\newcommand*{\kaishu}{\CJKfamily{zhkai}}
\newcommand*{\fangsong}{\CJKfamily{zhfs}}

\renewcommand*\contentsname{目录}

\usepackage{cchess}
\usepackage[margin=3.5cm]{geometry}
\usepackage{hyperref}

\title{机器人象棋程序手册}
\author{顾聪}
\date{2011年7月}
\begin{document}
\maketitle
\tableofcontents


\section{概述}
此手册的目的是提供使用和理解此计算机中国象棋程序所必需的信息。此组程序
所做的工作包括:人机象棋对弈流程跟踪、读取棋盘(串口)、发送机器手控制信
息(串口)、象棋运算引擎(eleeye\_xb)以及一个学习棋盘和棋子的辅助程序。
以上各部分可独立调试运作,也能相互配合完成整个任务。

此程序设计运行于32位Linux操作系统 \footnote{一个自由,免费,开放源代码
  的Unix操作系统内核。包含一定组件、开发工具、应用程序的Linux操作系统称
  为一个``发行版''(distro),常见的发行版有Fedora、Ubuntu、Archlinux等。
  各发行版如何安装系统、添加软件等问题超出本文范筹,请查阅所选发行版的
  文档。}上。除串口读写程序尚未正确移植外都可通过 Cygwin
\footnote{在Microsoft Windows下实现的Unix系统调用API。同时移植有许
  多Unix下的开发工具和应用程序。}重新编译,从而移植到 Microsoft
Windows 下。请注意,若需要修改和编译此程序,请在发行版或Cygwin中安
装GNU工具链\footnote{主要包括GCC(GNU Compiler Collection,含有C与C++编
  译器)、GNU make等}。

此程序以GNU通用公共许可证(GNU GPL)\footnote{官方文
  本:\url{http://www.gnu.org/licenses/gpl.html}。}发布。

最新的源代码可在\url{http://github.com/gucong/robotxq}找到。

\section{用语与表示法}
以下描述程序中存储和交换中国象棋相关数据时所使用的用语和表示法。这是使
用和了解计算机象棋程序的先决条件。

\subsection{位置}
借用国际象棋使用的坐标方式表示棋盘位置。用英文字母a~i表示纵线(列),用数
字0~9表示横线(行),二者合成表示棋盘上的位置,如a1、c9等,如图。

\smallboard
\begin{position}
\end{position}

\subsection{棋子}
红方棋子用大写英文字母表示,黑方棋子用小写英文字母表示,所用字母如下。

\begin{tabular}{l l}
\textpiece{k}\textpiece{K} (KING)   &用K,k表示;\\
\textpiece{g}\textpiece{G} (ADVISOR)&用A,a表示;\\
\textpiece{b}\textpiece{B} (BISHIP) &用B,b表示;\\
\textpiece{n}\textpiece{N} (KNIGHT) &用N,n表示;\\
\textpiece{r}\textpiece{R} (ROOK)   &用R,r表示;\\
\textpiece{c}\textpiece{C} (CANNON) &用C,c表示;\\
\textpiece{p}\textpiece{P} (PAWN)   &用P,p表示;\\
\end{tabular}

\subsection{局面}
\label{fen}
借用国际象棋中的FEN\footnote{Forsyth-Edwards表示法(Forsyth-Edwards
  Notation)}来表示一个局面。实现任一局面由一行简短的文本来表示,方便局
面信息的传递。FEN由6个段组成,由空格分隔。由于中国象棋与国际象棋的区别,
一般只需要使用FEN的前两项。以下是初始局面的FEN表示:

\begin{verbatim}
rnbakabnr/9/1c5c1/p1p1p1p1p/9/9/P1P1P1P1P/1C5C1/9/RNBAKABNR w - - 0 1
\end{verbatim}

第一段描述棋子位置。由斜杠(/)分隔的每一小段描述棋盘上的一行,依次为
行9至行0。每一行中,从列a至列i,遇棋子则写代表棋子的英文字母,遇空位则
写相连空位的个数。

第二段描述轮到哪一方走子,w表示红方,b表示黑方。

第三、第四段在中国象棋中不适用,填短横杠(-),第五段是可逆半回合数,第六
段是总回合数。在这里,缺失三、四、五、六段的,也认为是合法的FEN。


\smallboard
\begin{position}
  \piece{a}{0}{r} \piece{i}{0}{r}
  \piece{b}{0}{n} \piece{h}{0}{n}
  \piece{c}{0}{b} \piece{g}{0}{b}
  \piece{d}{0}{g} \piece{f}{0}{g}
  \piece{e}{0}{k}
  \piece{b}{2}{c} \piece{h}{2}{c}
  \piece{a}{3}{p} \piece{c}{3}{p} \piece{e}{3}{p} \piece{g}{3}{p} \piece{i}{3}{p}

  \piece{a}{9}{R} \piece{i}{9}{R}
  \piece{b}{9}{N} \piece{h}{9}{N}
  \piece{c}{9}{B} \piece{g}{9}{B}
  \piece{d}{9}{G} \piece{f}{9}{G}
  \piece{e}{9}{K}
  \piece{b}{7}{C} \piece{h}{7}{C}
  \piece{a}{6}{P} \piece{c}{6}{P} \piece{e}{6}{P} \piece{g}{6}{P} \piece{i}{6}{P}
\end{position}

\subsection{走子}
使用坐标表示一个着法,格式为起始位置加目标位置。例如,初始局面下红
方``炮二平五''表示成``b2e2''。吃子着法同样用主动子的起始位置加目标位置
表示。例如,黑方接着``炮八进七''就表示成``h7h0''。

\smallboard
\begin{position}
  \piece{a}{0}{r} \piece{i}{0}{r}
  \piece{b}{0}{n} \piece{h}{0}{n}
  \piece{c}{0}{b} \piece{g}{0}{b}
  \piece{d}{0}{g} \piece{f}{0}{g}
  \piece{e}{0}{k}
  \piece{e}{2}{c} \piece{h}{2}{c}
  \piece{a}{3}{p} \piece{c}{3}{p} \piece{e}{3}{p} \piece{g}{3}{p} \piece{i}{3}{p}

  \piece{a}{9}{R} \piece{i}{9}{R}
  \piece{b}{9}{N} \piece{h}{9}{N}
  \piece{c}{9}{B} \piece{g}{9}{B}
  \piece{d}{9}{G} \piece{f}{9}{G}
  \piece{e}{9}{K}
  \piece{b}{7}{C} \piece{h}{7}{C}
  \piece{a}{6}{P} \piece{c}{6}{P} \piece{e}{6}{P} \piece{g}{6}{P} \piece{i}{6}{P}
\end{position}
\smallboard
\begin{position}
  \piece{a}{0}{r} \piece{i}{0}{r}
  \piece{b}{0}{n} \piece{h}{0}{C}
  \piece{c}{0}{b} \piece{g}{0}{b}
  \piece{d}{0}{g} \piece{f}{0}{g}
  \piece{e}{0}{k}
  \piece{e}{2}{c} \piece{h}{2}{c}
  \piece{a}{3}{p} \piece{c}{3}{p} \piece{e}{3}{p} \piece{g}{3}{p} \piece{i}{3}{p}

  \piece{a}{9}{R} \piece{i}{9}{R}
  \piece{b}{9}{N} \piece{h}{9}{N}
  \piece{c}{9}{B} \piece{g}{9}{B}
  \piece{d}{9}{G} \piece{f}{9}{G}
  \piece{e}{9}{K}
  \piece{b}{7}{C}
  \piece{a}{6}{P} \piece{c}{6}{P} \piece{e}{6}{P} \piece{g}{6}{P} \piece{i}{6}{P}
\end{position}

\subsection{一维形式}
\label{onedim}
程序内部有时使用一维方式存储棋子位置。先从左至右,再从下至上,位
置0代表a0,位置1代表b0,\dots,位置8代表i0,位置9代表a1,\dots,依此类
推,直到位置89代表i9。各位置若有棋子,则填入代表棋子的英文字母,若为空
位,则填入下划线(\_)。另外,学习棋盘时也按此顺序。

\section{关于象棋引擎}
象棋引擎(简称引擎)是专门负责象棋着法思考的程序,可与图形界面等程序(简称
界面)通过标准输入输出(stdio)管道相连进行通信。robotxq实际上也是一个针对
物理棋盘的界面。象棋引擎为了与界面通信,需要符合某种协议,达到不同引擎
在同一界面之下互相替换、同台竞技的目的。常见的象棋通信协议有两种,一种
是ucci协议\footnote{官方文
  档(中文):\url{http://www.xqbase.com/protocol/cchess_ucci.htm}}(类似
于国际象棋的uci协议,但有一定区别);另一种
为xboard协议\footnote{也称winboard协议,官方文
  档(英文):\url{http://home.hccnet.nl/h.g.muller/engine-intf.html}},
中国象棋与国际象棋均适用。这两种协议在引擎和界面的分工上有很大的区
别。ucci引擎只负责就给定的局面思考一步着法;而xboard引擎同时在内部跟踪
棋局的进展,大多还能判断对手着法的合法性等。这些额外的功能对一个象棋引
掣来说是顺手能实现的,而在ucci协议中都是由界面程序从头做起。可
见,xboard引擎更像是一个完整的象棋游戏。而ucci引擎也有好处,它保证了无
后效性,也杜绝了引擎之间对规则的不同理解产生的问题(事实上,中国象棋长打
等规则非常复杂)。

在个这里应用里,xboard协议更方便界面的编写。也能更好地保持象棋和国际象
棋在界面编写上的统一。所以,robotxq选择使用xboard协议。

此程序默认的引擎Eleeye(Elephant Eye,\url{http://www.xqbase.com})本身是
一个的ucci协议引擎。通过对其修改,使其成为了原生的xboard协议引擎,称
为eleeye\_xb。eleeye\_xb引擎实现了绝大部分xboard协议的命令,具备完整的
着法合法性检查。暂时缺少的功能是利用对手时间思考(ponder)。

另外,通过适配
器ucci2wb(\url{http://home.hccnet.nl/h.g.muller/XQadapters.html}),任
一ucci引擎可以转化为xboard引擎,但转化后的引擎不具着法合法性检查。

\section{安装与初次使用}
命令的基本使用可参照附录\ref{app1}。
%%% sysconfdir

\section{程序组成}
为了灵活地在运行时选择组件,整个程序由数个可独立运行的部分合作运行,组
件的选择只需在起动主程序时给出相应的命令行参数即可。组件包括:人机象棋
对弈流程跟踪(robotxq)、象棋引擎(eleeye\_xb)、读取棋盘(io\_board)、控制
机器手(io\_hand)。另有独立的辅助程序:查看棋盘(catboard),学习棋盘和棋
子(learn\_brd)。

\subsection{配置文件:xq.fen与xq.brd}
\label{conf}
(示例的配置文件在源码目录中的conf文件夹中)

配置文件都是纯文本文件,可用文本编辑软件进行修改。

xq.fen保存起始局面列表。开始时的局面从中随机选取。其格式为:任意多
个FEN格式串(见\ref{fen}),每个占一行;``end'',独占一行;之后的内容不会
被读取,可以用来写注释或存放暂时不用的开始局面。

xq.brd保存棋盘以及棋子的对应表。其可用棋盘和棋子学习程序learn\_brd生成,
也可手动对学习中的错误进行修改。第一行为棋盘对应表,其中的数字按一维形
式(见\ref{onedim})排列,数字表示该棋盘位置在原始数据中的排列位置。第二
行和第三行为棋子对应表。第三行中的数字为棋子的ID号,按第二行的中棋子的
顺序排列。

各程序若要用到这些配置文件,则按如下顺序调取:若程序参数指定,则用此文
件;否则,若 \~\ /.robotxq 文件夹下存在,则用此文件;否则使用编译时指定的
系统配置文件夹(sysconfdir)下的预装的配置文件。

所以,用户可以把需要的配置文件保存
到\~\ /.robotxq/xq.fen和\~\ /.robotxq/xq.brd,这样,各程序就会自动优先调取它
们,而不必每次都在参数中指定。

\subsection{人机象棋对弈流程跟踪(主程序):robotxq}
人机象棋对弈流程跟踪程序的帮助信息如下:
\begin{verbatim}
$ robotxq -h
Usage: robotxq [OPTION]... DEVICE
robotxq -- Robot plays Xiangqi
OPTION:
  -h            display this help and exit
  -e ENGINE     use ENGINE as the engine program
  -f FEN_FILE   use positions in FEN_FILE as start positions
  -r READER     use READER as the program to read the board in
  -b BRD_FILE   use BRD_FILE as the board configuration file
\end{verbatim}

DEVICE用来指定读取棋盘的串口设备,见\ref{dev}。

ENGINE选项用来指定象棋运算引擎,默认为eleeye\_xb。

READER选项用来指定棋盘读取程序。默认为io\_board。调用此读取程序时,主程
序会自动添加所指定串囗设备DEVICE。

FEN\_FILE选项用来指定起始局面列表文件,BRD\_FILE选项用来指定棋盘棋子对
照表文件。它们的默认位置见\ref{conf}中的说明。

%%% 使用...

\subsection{象棋引擎:eleeye\_xb}
\subsection{棋盘读取程序:io\_board}
\subsection{控制机器手:io\_hand}
\subsection{查看棋盘:catboard}
\subsection{学习棋盘和棋子:learn\_brd}















\clearpage
\appendix
\section{Unix命令行的基本使用}
\label{app1}
\subsection{设备文件}
\label{dev}
%%% 介绍 device node
%%% 介绍 文件系统 / ~ 

\end{document}